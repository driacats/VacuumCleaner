\label{sec:introduction}
The scenario of the cleaning robot is often used in textbooks as an example to present intelligent agents and their basic principles; one need only cite ``Artificial Intelligence: A Modern Approach''\cite{DBLP:books/aw/RN2020} by S. Russell and P. Norvig, who use it as early as the second chapter as an introduction to the agent concept.
The case study is simple, a robot vacuum cleaner is in a grid environment and has to clean the whole space by having motion and cleaning as actions. There are then various versions related to map knowledge that obviously influence the agent's behavior. In the most basic versions the agent finds out whether a cell contains dirt or not only when it is standing on it, in others it knows where the dirt is on the entire map and must find the best way to complete the task.

The robot vacuum cleaner in these examples reasons about a space being discretized and does so very low-level, deciding what individual steps to take and in what direction.
Also, these paths reduce the agents' rotational degrees of freedom to the four directions above, below, right and left.
This discretization is certainly necessary if one thinks that an agent moving through the space has to reason not so much about \textit{what} it has to do but more about \textit{how} it has to do it and the individual steps that lead it to achieve its goal.
While this may be true for some type of agents, it is certainly reductive for logical agents based on the Belief-Desire-Intention paradigm, which instead have their strength precisely in abstraction and in delegating the management of the environment to external artifacts.

Moreover, in the current state of the art, most simulation software and game engines carry much more advanced path finding technologies than a simplistic discrete representation of the world.
Godot\cite{godot} implements \href{https://docs.godotengine.org/en/stable/tutorials/navigation/navigation_using_navigationpaths.html}{NavigationPaths}, Unity\cite{unity} has an entire \href{https://docs.unity3d.com/2021.3/Documentation/Manual/Navigation.html}{Navigation System} that implements path finding, and even Unreal Engine\cite{unreal} implements \href{https://dev.epicgames.com/documentation/en-us/unreal-engine/basic-navigation-in-unreal-engine}{Navigation Mesh} that is used by agents to move through space.
All these tools also implement the handling of steps, ascents, descents, and other situations that in a logical implementation should be considered as separate instances and handled properly.

Even in the field of robotic research, we can find multiple approaches to finding optimal paths that consider space in the continuum, finding detailed solutions that do not simply involve left, right steps with predetermined rotations but true paths to be followed with 360-degree freedom of rotation.
A complete overview of the state of the art from this point of view can be found in the survey\cite{Karur2021ASO} by K. Karur et al. on path planning for mobile robots.
The paper makes a first major division between global algorithms in which the agent has complete knowledge of the environment and local algorithms in which, on the other hand, it knows only a neighborhood of its location and must therefore update its path as it moves.
The algorithms presented move from discrete to continuous space, demonstrating that here, too, technology has moved beyond a cellular representation of the world.

In the research branch on BDI agents, on the other hand, we find another survey exploring recent spatial reasoning techniques for BDI agents\cite{cilc}.
In the approaches presented, agents reason about where they are, often discretizing, but still always deciding on individual movement directly by the agent.

Some works reason directly about Cartesian coordinates, others about the four directions (front, behind, right, left), and others with structures borrowed from other areas of research (such as Entity-Relation diagrams).
Of all the approaches, the most widely used is the Region Connection Calculus, which will be presented in the Design and Implementation section since our implementation also makes use of it.

The paper we present seeks a solution to the problem of robotic vacuuming that takes into account the novelties on the simulation and robotics side, going to use state-of-the-art path planning tools in the continuum, without limiting the robot's degrees of freedom, reasoning about the regions to be explored using a logical agent that makes the implementation furthermore explicable and transparent, as well as enhancing the programming paradigm.
In this early work the agent has a comprehensive knowledge of the environment and where the dirt is.
An exploratory version of the project is already in the works in which the agent starts without a map and has to create it from scratch.
A brief mention of what has been done and future developments will be in the Future Work section.