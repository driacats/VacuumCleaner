\label{sec:introduction}
The cleaning robot is a landmark case study in the world of multi-agent systems, allowing us to think about spatiality with a precise and well-defined goal: to remove all dirt.
The classic example involves a grid environment of which some cells are dirty, and an agent located on it that must find and clean all the dirty cells knowing only how to take a step in one of the four directions and remove the dirt.
With the rapid evolution of game engines, however, the "step-by-step" approach seems to be becoming limiting: we no longer need to think this way; we can abstract more.

In this paper, we propose a cleaning robot that explores space by reasoning about the continuum and leaving the precise management of motion to the environment.