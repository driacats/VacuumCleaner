We presented a framework in which a BDI agent controls a robot vacuum cleaner with a precise division of responsibilities, giving logical task management to the agent and physical management of movement and cleaning to the robot.
Going through the key theoretical concepts, we showed how the environment must be studied and designed for this kind of approach, then going on to see the actual implementation of all components with JaCaMo and Godot.
The implementation aims not to discretize the robot's movements by forcing them within some predetermined logical actions but by choosing the actions logically, implementing them in the continuum.

The current implementation has many limitations, the two most important being the management of regions on the virtual environment side and the insertion of objects into the logical map.
In Godot, the region should be implemented as an Area3D, looking for the nearest location without having to reach the center or a specific point.
Agent side, the map should also contain objects and objects should be considered, thinking of a scenario where they are most meaningful.

Future work will therefore go in these directions.
In addition to this, an exploratory version of the framework will be implemented in which the agent faces the challenge of cleaning without knowing the map and without knowing where the dirty areas are.
The real challenge in this version will be to figure out how to subdivide the walkable regions, what are the criteria that make a region stand on its own or not.